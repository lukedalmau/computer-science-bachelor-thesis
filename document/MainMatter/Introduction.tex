\chapter*{Introducción}\label{chapter:introduction}
\addcontentsline{toc}{chapter}{Introducción}

La población sordomuda es una minoría no despreciable, en cuanto a desempeño social, a la cual se debe prestar principal atención. Muchas de estas personas sordomudas esconden un verdadero potencial humano el cual no pueden desarrollar en su máxima capacidad debido a los impedimentos sociales en que incurren, a causa del problema comunicativo mayoritariamente. El autor quiere con este trabajo lograr disminuir esa brecha comunicativa lo máximo posible para alcanzar un mundo mejor, en el que ser sordomudo no represente una limitación en el proceso de la comunicación social, ni en su desarrollo individual.

Alrededor de 72 millones, de las 8 mil millones de personas que vivimos en este mundo, son personas sordas que utilizan lenguas de señas para comunicarse, según la Federación Mundial de Sordos (WFD por sus siglas en inglés)[\cite{without_sign_language_deaf_people_are_not_equal_2019}].

Aproximadamente más del $80\%$ de esos 72 millones viven en países en desarrollo, donde las autoridades locales no promueven ni apoyan políticas o acciones para suplir sus necesidades o incluirlos en la sociedad.

En Cuba, la Asociación Nacional de Sordos de Cuba (ANSOC) cuenta con alrededor de 25 mil asociados y aproximadamente 400 intérpretes registrados, de un total de 11 millones aproximadamente de habitantes, según especifica Yoel Moya, Subdirector de Investigación del Centro Nacional de Superación y Desarrollo del Sordo (CENDSOR). Sería asombroso imaginar la cantidad de tiempo, dinero y preparación que se ahorraría el país si se tuviera, en tiempo real, una traducción de texto a lengua de señas(análogamente de audio a lengua de señas se puede realizar con un paso extra de llevar de audio a texto).

Como es una proporción abrumadora, las cifras del número de personas sordomudas representan menos del $1\%$ del total de personas, tanto a nivel mundial, como nacional. Por tanto sería prácticamente imposible para una persona sordomuda llegar a entenderse con personas en una actividad del día a día.

Una solución es lograr traducir del lenguaje natural a la lengua de señas. La solución que se propone es generar lengua de señas en forma de video con un avatar para que las personas sordomudas que sepan lengua de señas puedan entenderse con la otra parte de la población que no habla dicha lengua. De ser lo suficientemente preciso puede servir además a modo de tutorial para aquellas personas que no sepan lengua de señas y puedan aprender mediante la repetición de los gestos del avatar.


En la actualidad este problema de la comunicación con personas sordomudas es abarcado por disímiles campos de investigación, como la interpretación automática de las lenguas de señas, la generación de avatares señantes, entre otros. Si bien este proceso
se puede entender como una traducción interlingual, las características  específicas de las lenguas de señas hacen que su traducción e interpretación constituyan un proceso
no tan sencillo que involucra componentes visuales. Es por dichos componentes que intervienen además varios
campos de investigación como el procesamiento de lenguaje natural y la visión por
computadora para asistir al entendimiento y procesamiento correcto de las lenguas de señas [\cite{leynier-lsc-2021}].

\section{Problemática}

En lo que se refiere a la generación de avatares para la lengua de señas existen avances 
significativos a nivel mundial. La Lengua de Señas Cubana es diferente de otras lenguas
de señas utilizadas en el mundo y de cualquier idioma hablado conocido. Nuestra lengua se 
utiliza en muchos países con algunas modificaciones en cada país pero somos capaces de comunicarnos 
entre todos. Por el contrario la lengua de señas cubana es única y por tanto Cuba es el único país responsable, en realizar estudios, investigaciones y herramientas para 
impulsar su aceptación.

Aunque en los últimos años en el país se han impulsado las tecnologías de la información y las 
telecomunicaciones, no se ha utilizado estos avances en pos de la generación de avatares para lograr 
una mayor inclusión de la sociedad hipoacústica . Es debido a ello que se hace importante
fomentar estudios e implementar plataformas que apoyen e inciten a una mejor inserción
en la sociedad de las personas que dependan de la Lengua de Señas Cubana para
comunicarse.

\section{Objetivos del Trabajo}

 
\subsection{Objetivo general}
\begin{itemize}
\item Proponer un modelo para la generación automática de avatares gráficos para la traducción
automática de la lengua de señas cubana

\end{itemize}

\subsection{Objetivos específicos}
\begin{itemize}
\item Estudiar el estado del arte en la generación de avatares para las lenguas de señas
\item Identificar el conjunto de símbolos y acciones que se conocen por investigaciones previas de la lengua de señas cubana
\item Proponer un modelo para la generación de avatares gráficos en la lengua de 
señas
\item Proponer un modelo basado en aprendizaje de máquina que permita dotar a los avatares de rasgos que permitan la identificación del hablante en relación al avatar
\item Realizar un conjunto de experimentos que permitan probar la viabilidad de su propuesta
\end{itemize}

\section{Propuesta de solución}
Como método de solución se propone un sistema por partes, utilizando redes
neuronales recurrentes, en específico un modelo basado en Memorias Grandes de Corto Plazo (LSTM por sus siglas en inglés) y además, de utilizar capas convolucionales transpuestas (1DConvT) para el incremento de dimensionalidad.
Se utiliza el corpus disponible por trabajos anteriores en el tema, se le hace una limpieza a las palabras y se equiparan los frames de la secuencia de puntos del corpus. Además, se utiliza un embedding de palabra entrenado en el idioma español para aportarle generalización al modelo para luego usar dicho vector como entrada del modelo de las capas anteriormente mencionadas. Luego se procesa el resultado obtenido en forma de matriz para transformarlo a la dimensión que permita efectuar el dibujo. 


\section{Estructura del trabajo}
Este trabajo se divide en 6 partes. El capítulo introductorio plantea la
problemática y sus consecuencias en el contexto social correspondiente, describiendo posteriormente, los objetivos y un breve resumen de la
propuesta del trabajo, terminando con la explicación de su estructura. El capítulo
1 aborda un estudio realizado sobre el estado del arte del campo, mostrando varios
conceptos e ideas que se aplican en los capítulos siguientes. El capítulo 2 presenta el diseño
del sistema con el modelo de capas recurrentes y convolucionales propuesto para la generación automática de avatares para la lengua de señas cubana. En el capítulo 3 se explica la implementación de un prototipo del modelo primario
propuesto, los experimentos para validar la propuesta y los resultados obtenidos. Seguidamente se exponen las Conclusiones del trabajo y un grupo de Recomendaciones para investigaciones y desarrollos futuros. Culmina con la lista de las referencias bibliográficas utilizadas para su elaboración.