\chapter*{Introducción}\label{chapter:introduction}
\addcontentsline{toc}{chapter}{Introducción}

La población sordomuda es un sector de la población no despreciable, en cuanto a desempeño social, a la cual se debe prestar gran atención. Muchas de estas personas sordomudas esconden un verdadero potencial humano el cual no pueden desarrollar en su máxima capacidad debido a los impedimentos sociales en que incurren, a causa del problema comunicativo mayoritariamente. Con este trabajo se quiere dar pasos para disminuir esa brecha comunicativa lo máximo posible para alcanzar un mundo mejor, en el que ser sordomudo no represente una limitación en el proceso de la comunicación social, ni en su desarrollo individual.

Alrededor de 72 millones, de las 8 mil millones de habitantes en este mundo, son personas sordas que utilizan lenguas de señas para comunicarse, según la Federación Mundial de Sordos (WFD, por sus siglas en inglés)\brackcite{without_sign_language_deaf_people_are_not_equal_2019}. Aproximadamente, más del $80 \%$ de esos 72 millones viven en países en desarrollo, donde las autoridades locales no promueven ni apoyan políticas o acciones para suplir sus necesidades o incluirlos en la sociedad.

En Cuba, la Asociación Nacional de Sordos de Cuba (ANSOC) cuenta con alrededor de 25 mil asociados y aproximadamente 400 intérpretes registrados, de un total de 11 millones aproximadamente de habitantes, según especifica Yoel Moya, Subdirector de Investigación del Centro Nacional de Superación y Desarrollo del Sordo (CENDSOR). Sería asombroso imaginar la cantidad de tiempo, dinero y preparación que se ahorraría el país si se pudieran generar, con el vocabulario existente, avatares para las señas conocidas.

Como es una proporción abrumadora, las cifras del número de personas sordomudas representan menos del $1 \%$ del total de personas, tanto a nivel mundial, como nacional. Por tanto, sería prácticamente imposible para una persona sordomuda llegar a entenderse con cualquier persona en una actividad del día a día.

 La solución que se propone es generar lengua de señas en forma de video con un avatar para que, en un futuro, las personas sordomudas que sepan lengua de señas puedan entenderse con la otra parte de la población que no habla dicha lengua. De ser lo suficientemente preciso puede servir, además, a modo de tutorial para aquellas personas que no sepan lengua de señas y puedan aprender mediante la repetición de los gestos del avatar.
 
 En trabajos anteriores se refieren a esta problemática abordando el tema de la comunicación de la siguiente forma:

\begin{quote}


 ``En la actualidad este problema de la comunicación con personas sordomudas es abarcado por disímiles campos de investigación, como la interpretación automática de las lenguas de señas, la generación de avatares señantes, entre otros. Si bien este proceso se puede entender como una traducción interlingual, las características  específicas de las lenguas de señas hacen que su traducción e interpretación constituyan un proceso no tan sencillo que involucra componentes visuales. Es por dichos componentes que intervienen además varios campos de investigación como el procesamiento de lenguaje natural y la visión por computadora para asistir al entendimiento y procesamiento correcto de las lenguas de señas.''
\begin{flushright}
Gutiérrez$-$González\brackcite{leynier-lsc-2021}
\end{flushright}
\end{quote}

\section*{Problemática}

En lo que se refiere a la generación de avatares para la lengua de señas existen avances 
significativos a nivel mundial. La Lengua de Señas Cubana es diferente de otras utilizadas en el mundo y de cualquier idioma hablado conocido. El Español se 
utiliza en muchos países con algunas modificaciones regionales pero que no impide la capacidad de comunicarnos entre todos. Por el contrario la lengua de señas cubana es única y, por tanto, Cuba es el único país responsable, en realizar estudios, investigaciones y herramientas para 
acelerar su aceptación.

Aunque en los últimos años en el país se han impulsado las tecnologías de la información y las 
telecomunicaciones, no se han utilizado suficientemente estos avances en pos de la generación de avatares para lograr 
una mayor inclusión de la sociedad hipo-acústica. Es debido a ello que se hace importante
fomentar estudios e implementar plataformas que apoyen e inciten a una mejor inserción
en la sociedad de las personas que dependan de la Lengua de Señas Cubana para
comunicarse.

\section*{Objetivos del Trabajo}

 
\subsection*{Objetivo general}
\begin{itemize}
\item Proponer un modelo que contribuya a la generación automática de avatares gráficos para la traducción
automática de la lengua de señas cubana

\end{itemize}

\subsection*{Objetivos específicos}
\begin{itemize}
\item Estudiar el estado del arte en la generación de avatares para las lenguas de señas
\item Identificar el conjunto de símbolos y acciones que se conocen por investigaciones previas de la lengua de señas cubana
\item Proponer un modelo para las primeras etapas en la generación de avatares gráficos en la lengua de 
señas
\item Realizar un conjunto de experimentos que permitan probar la viabilidad de su propuesta
\end{itemize}

\section*{Propuesta de solución}
Como método de solución se propone un sistema por partes, utilizando, primeramente, un preprocesamiento de los datos existentes obtenidos en investigaciones previas garantizando el correcto etiquetado de las señas. Dicho conjunto de datos es guardado en un corpus de extensión JSON para su posterior uso. Se crea una estructura para guardar la representación de dichas señas (tokens), lo que permite instanciarlos, manejarlos y concatenarlos, garantizando su correcta interpolación y su posterior exportación a vídeo o graficado, utilizando motores gráficos.  

\section*{Estructura del trabajo}
Este trabajo se divide en 6 partes. El capítulo introductorio plantea la problemática y sus consecuencias en el contexto social correspondiente, describiendo, posteriormente, los objetivos y un breve resumen de la propuesta del trabajo, finalizando con la explicación de su estructura. El capítulo 1 aborda un estudio realizado sobre el estado del arte del campo, mostrando varios conceptos e ideas que se aplican en los capítulos posteriores. El capítulo 2 presenta el diseño de la propuesta basada en distintos tipos de interpolación. En el capítulo 3 se explica la implementación de un prototipo del modelo propuesto, los experimentos para validar la propuesta y los resultados obtenidos. Seguidamente, se exponen las conclusiones del trabajo y un grupo de recomendaciones para investigaciones y desarrollos futuros. Culmina con la lista de las referencias bibliográficas utilizadas para su elaboración.