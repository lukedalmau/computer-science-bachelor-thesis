\begin{opinion}
    Es una responsabilidad de la sociedad fomentar y establecer sistemas que permitan a las personas con discapacidad vivir plenamente. Las personas con limitaciones en el habla y en su capacidad auditiva tienen muchas dificultades en su comunicación con otras personas. En este contexto, las tecnologías de la información y las comunicaciones pueden proporcionar soluciones. Sin embargo, en Cuba no se han desarrollado muchas investigaciones y/o desarrollos en este campo. Por ello, el Grupo de Investigación en Inteligencia Artificial de la Universidad de La Habana comenzó a trabajar en una línea de investigación cuyo objetivo final es crear una plataforma útil para mejorar la comunicación de las personas con discapacidad auditiva y del habla mediante el uso de la Lengua de Señas Cubana.

En este sentido, el trabajo de diploma ''Generación gráfica de señas de la Lengua de Señas Cubana: una primera aproximación'' del estudiante Luis Enrique Dalmau Coopat es una continuación de la línea de investigación antes mencionada. Como parte de este trabajo se profundiza en el estado del arte de diversos campos relacionados con la escritura de señas, con énfasis en su estructura y composición, el reconocimiento de señas, la tecnología de avatares, la interpolación de movimiento y los modelos generativos donde se observó un aumento en la aceptación de los modelos que utilizan inteligencia artificial y aprendizaje de máquina. Se propone una estructura de métodos y clases para la interpolación y adición adecuadas de una secuencia de señas, logrando así una generación precisa de movimientos suaves y claros del esqueleto del avatar. La propuesta realizada para la graficación de señas de la Lengua de Señas Cubana, que utiliza el corpus generado por investigaciones anteriores, se concreta en un prototipo de clases que utiliza interpolación e integración numérica para resolver los problemas en cuestión. El resultado es un avatar capaz de ejecutar de manera suave y armoniosa los movimientos asociados a cada token de seña, incluso cuando se combinan señas de manera extensa.

Durante la realización del trabajo se enfrentaron disímiles problemáticas cómo la curación y normalización de los datos que evidencian la importancia de continuar trabajando en la recolección de más y mejores datos para futuras investigaciones. Además el estudiante propone ideas y pautas a seguir para desarrollos posteriores que se encarguen de generar avatares realistas utilizando modelos generativos.

El estudiante Luis Enrique Dalmau Coopat trabajó arduamente con una gran determinación, capacidad de trabajo y habilidades, tanto en la gestión como en el desarrollo e investigación. Además, su trabajo tiene un gran valor social por la utilidad de la línea de investigación continuada con este proyecto. Por todo ello, solicitamos que se le otorgue la mejor calificación posible y que, de este modo, obtenga el título de Licenciado en Ciencias de la Computación.\\

    \begingroup
  \centering
  \wildcard{Dr. Yudivián Almeida Cruz}
  \hspace{1cm}
  \wildcard{Lic. Leynier Gutiérrez González}
  \par
\endgroup
\end{opinion}