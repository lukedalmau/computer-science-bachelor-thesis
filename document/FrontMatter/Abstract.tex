\begin{resumen}

	Este trabajo sigue un enfoque nunca antes probado en el contexto de las investigaciones de la Lengua de Señas Cubana. Muchos países han avanzado en el aspecto de la producción de las lenguas de señas, probándose las redes convolucionales, las redes recurrentes, los transformadores y enfoques no autorregresivos, los cuales han brindado excelentes resultados en los respectivos países de dichas investigaciones. Se presenta una investigación acerca del uso de un modelo secuencia a secuencia basado en capas LSTM y capas deconvolucionales para la generación de avatares para la Lengua de Señas Cubana utilizando el corpus brindado por el CENDSOR.
	
\end{resumen}

\begin{abstract}

	This work follows an untested approach in the context of Cuban Sign Language research. Many countries have some advance in the aspects of the production of sign languages, testing convolutional networks, recurrent networks, transformers and non-autoregressive approaches, which have provided excellent results in the respective countries of those investigations. An investigation about the use of a sequence-to-sequence model based on LSTM layers and deconvolutional layers for the generation of avatars for the Cuban Sign Language using the corpus provided by CENDSOR is now presented.
	
\end{abstract}