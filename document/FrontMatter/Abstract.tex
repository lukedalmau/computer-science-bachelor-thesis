\begin{resumen}
Este trabajo sigue un enfoque nunca antes probado en el contexto de las investigaciones de la Lengua de Señas Cubana. Muchos países han avanzado en el aspecto de generación automática de avatares, probándose los transformadores y enfoques no autorregresivos, los cuales han brindado excelentes resultados en los respectivos países de dichas investigaciones pero se hace necesario tener un vasto conjunto de datos de párrafos u oraciones contra la correspondiente lengua de señas.  Se presenta una investigación acerca del uso de una propuesta basada en interpolación e integración numérica para la generación continua de avatares para la Lengua de Señas Cubana utilizando el corpus de señas aisladas brindado por el Centro Nacional para el Desarrollo y Superación del Sordo (CENDSOR) como los únicos datos disponibles.
	
\end{resumen}

\begin{abstract}
This work follows an approach never tested before in the context of Cuban Sign Language research. Many countries have advanced in the aspect of automatic generation of avatars, testing the transformers and non-autoregressive approaches, which have provided excellent results in the respective countries of those investigations but it is necessary to have big dataset of paragraphs or sentences corresponding to sign language. It is presented an investigation  about the use of a proposal based on interpolation and numerical integration for the continuous generation of avatars for the Cuban Sign Language using the corpus of isolated signs provided by the National Center for the Development and Overcoming of the Deaf (CENDSOR, by it's spanish meaning) as the only available data.
	
\end{abstract}