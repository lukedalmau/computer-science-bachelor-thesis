\begin{conclusions}
  La generación de avatares para la Lengua de Señas Cubana representa el principar objetivo del trabajo. Durante la investigación se generan nuevos métodos, estructuras y corpus que posibilitan una mejor comprensión de esta temática, como a su vez, amplia la base de conocimiento actualmente disponible.
 	Se introduce la problemática presente en el contexto social correspondiente y se profundiza en el estado del arte de diversos campos como la escritura de señas, haciendo enfásis en su estructura y composición,  traducción automática, tecnología de avatares, producción de lengua de señas y reconocimiento de la misma, detectándose un crecimiento de la aceptación de los modelos que utilizan inteligencia artificial y  aprendizaje de máquina.
 	Destacar entre estos aspectos que los métodos por partes obtuvieron mejores resultados cuando se contaban con pocos datos al respecto, mientras que los enfoques extremo a extremo utilizando transformers son los que mejores resultados han obtenido a nivel mundial utilizando bases de datos bastante ricas y abundantes. También referir a los únicos 2 trabajos encontrados que tratan de alguna forma u otra la Lengua de Señas Cubana.
 	Con todos estos ingredientes se escogió una propuesta de un modelo secuencia a secuencia, por partes, que utiliza un modelo secuencial de varias capas de redes recurrentes y  de redes deconvolucionales para el ensanchamiento de dimensiones. Esta propuesta para la generación de avatares para nuestra lengua de señas utilizando el corpus generado por anteriores investigaciones fue concretada en un prototipo que utiliza un modelo de incrustaciones de palabras entrenado en un enorme corpus de español para capturar de manera efectiva el significado de las frases del corpus para así generalizar más el modelo.
 	Se compararon las secuencias de imágenes obtenidas en las predicciones con las reales, obteniéndose buenos resultados a apreciación del autor. Dado que no existe un dominio del tema, la única falla grave es el hecho de que los dedos de las manos no se definen.  
 
\end{conclusions}
