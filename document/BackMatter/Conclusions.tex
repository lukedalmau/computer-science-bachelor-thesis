\begin{conclusions}

%%%%%%%%%%%%%%%%%%%%%
% Ajustarlas a los resultados obtenidos
%%%%%%%%%%%%%%%%%%%%%

  La generación de avatares para la Lengua de Señas Cubana representa el principal objetivo del trabajo. Durante la investigación se generan nuevos métodos, estructuras y corpus que posibilitan una mejor comprensión de esta temática. Se introduce la problemática presente en el contexto social correspondiente y se profundiza en el estado del arte de diversos campos como la escritura de señas, haciendo énfasis en su estructura y composición, reconocimiento de señas, tecnología de avatares, modelos generativos y  interpolación de movimiento, detectándose un crecimiento de la aceptación de los modelos que utilizan inteligencia artificial y  aprendizaje de máquina.
 	Se analizó el uso de los modelos generativos como una alternativa para la generación automática de avatares hiper-realistas de los cuáles se deja como recomendaciones para futuras investigaciones. Se indagó acerca de los únicos trabajos encontrados que tratan la Lengua de Señas Cubana. Se propuso una estructura de métodos y clases para la correcta interpolación y adición de una secuencia de señas, lográndose así una correcta generación de movimientos suaves y definidos del esqueleto del avatar. Esta propuesta para la generación de avatares para la Lengua de Señas Cubana, utilizando el corpus generado por anteriores investigaciones, fue concretada en un prototipo de clases que utilizan interpolación e integración numérica para resolver los problemas presentados.
 	Se obtuvo un avatar que consigue ejecutar de manera armónica los movimientos asociados a cada token de seña incluyendo cualquier combinación de los mismos sin importar cuán extensa sea.

\end{conclusions}
