\begin{recomendations}

%%%%%%%%%%%%%%%%%%%
% Retocarlas y ajustarlas un poco al tema
% Recomendar la continuacion de la investigacion
% Para la generacion de avatars humanos
%%%%%%%%%%%%%%%%%%%
Con este trabajo se ayuda a una nueva rama de investigación y desarrollo referente a la lengua de señas cubana, ya que forma un pilar esencial en cualquier sistema de traducción de lenguas de señas. Al ser este un trabajo novedoso en el campo de la generación de avatares señantes, se deja como pauta inicial para futuros trabajos y una guía del proceder ante los limitados datos disponibles.

Se sugiere:
\begin{itemize}
\item Estudiar la generación de archivos de captura de movimiento como BVH  y el cálculo de los vectores de rotación de brazos, manos y dedos.
\item Lograr transformar un vídeo de un avatar animado en una persona real haciendo uso de los modelos generativos, en especial el uso de los métodos de imagen a imagen (img2img)
\item Desarrollar alternativas con el uso de otros modelos generativos de imágenes para la humanización del avatar presentado.
\end{itemize}
 Se recomienda, a los investigadores de Cuba, lingüistas y especialistas de la lengua de señas y a toda la comunidad:
 \begin{itemize}
 \item Crear conciencia sobre la importancia y necesidad de este tipo de estudios y herramientas para el correcto desarrollo de la comunidad sordo-muda cubana
 \item Trabajar unánimemente en pos de lograr un corpus correctamente anotado y tanto extenso, como generalizado, para así obtener una herramienta de utilidad para que, en manos de los hipoacúsicos, sea la llave para romper las barreras comunicativas que les impiden desenvolverse y desarrollarse en nuestro país actualmente. 
\end{itemize}  
\end{recomendations}
