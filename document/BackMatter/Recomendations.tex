\begin{recomendations}
    Con este trabajo se inicia una nueva rama de investigación y desarrollo referente a la producción de lengua de señas cubana.
    Al ser este un trabajo novedoso en el campo de la producción de señas, se deja como pauta inicial para futuros trabajos y una guía del proceder ante los limitados datos disponibles. Por lo que se seguirá trabajando en cumplirse los objetivos sin resolver en este trabajo y en el desarrollo de otros modelos para compararse los resultados.
   Se recomienda, a los investigadores de Cuba y a toda la comunidad, el crear conciencia sobre la importancia y necesidad de este tipo de estudios y herramientas para el correcto desarrollo de la comunidad sordo-muda cubana.
   Se incita a los desarrolladores cubanos, lingüistas y especialistas del lenguaje de señas a trabajar unánimemente en pos de lograr un corpus correctamente anotado y tanto extenso, como generalizado, para así obtener una herramienta de utilidad para que, en manos de los hipoacúsicos, sea la llave para romper las barreras comunicativas que les impiden desenvolverse y desarrollarse en nuestro país actualmente. 
\end{recomendations}
